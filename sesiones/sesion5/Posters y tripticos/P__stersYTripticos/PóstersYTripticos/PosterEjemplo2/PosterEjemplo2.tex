\documentclass{beamer}
  \usepackage[orientation=portrait,size=a0,scale=1.4,debug]{beamerposter}                      

  \mode<presentation> {  
    \usetheme{Warsaw}
  }
  
  \usefonttheme[onlymath]{serif}
  \boldmath
  
  \usepackage[english]{babel}
  \usepackage[utf8]{inputenc}
  \usepackage{amsmath,amsthm, amssymb, latexsym}
   
  \title[Posters]{\VERYHuge Haciendo un poster en \LaTeX}
  \author[Alex]{\huge Alexander Borb\'on Alp\'izar}
  \institute[ITCR]{\Large Instituto Tecnol\'ogico de Costa Rica}
  \date{Enero-Febrero, 2013}
  
  \begin{document}
  \begin{frame}[plain]{} 
  	\maketitle
    \vfill
    \begin{block}{\large Fontsizes}
      \centering
      {\tiny tiny}\par
      {\scriptsize scriptsize}\par
      {\footnotesize footnotesize}\par
      {\normalsize normalsize}\par
      {\large large}\par
      {\Large Large}\par
      {\LARGE LARGE}\par
      {\veryHuge veryHuge}\par
      {\VeryHuge VeryHuge}\par
      {\VERYHuge VERYHuge}\par
    \end{block}
    \vfill
    \begin{columns}[t]
      \begin{column}{.48\linewidth}
        \begin{block}{Introducci\'on}
          En este art\'iculo...
        \end{block}
      \end{column}
      \begin{column}{.48\linewidth}
        \begin{block}{Secci\'on 2}
          \begin{itemize}
          \item item 1 y $\int f(x) dx$
          \item item 2
          \end{itemize}
        \end{block}
        \begin{block}{Secci\'on 3}
          \begin{itemize}
          \item item 1
          \item item 2
          \end{itemize}
          $\int f(x) dx$
        \end{block}
      \end{column}
    \end{columns}
  \end{frame}
\end{document}