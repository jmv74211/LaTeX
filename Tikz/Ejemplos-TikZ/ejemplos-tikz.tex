\documentclass[]{scrartcl}
\usepackage[spanish]{babel}
\usepackage[utf8]{inputenc}

\usepackage{tkz-graph}
\thispagestyle{empty}

%opening
\title{Algunos ejemplos con TikZ}
\author{Pedro González Rodelas}

\begin{document}

\maketitle

\begin{abstract}

\end{abstract}

\section{Ejemplos simples}


\begin{figure}[!h]
	\centering
\begin{tikzpicture}
\GraphInit[vstyle=Normal]
\SetGraphUnit{2}
\begin{scope}[rotate=-135]
\Vertices{circle}{A,B,C,E}
\end{scope}
\NOEA[unit=1.414](E){D}
\Edges(A,B,E,D,C,E,A,C,B)
\end{tikzpicture}
\end{figure}

El mismo ejemplo, modificando algo el estilo de los nodos.

\begin{figure}[!h]
	\centering
\begin{tikzpicture}
\GraphInit[vstyle=Art]
\begin{scope}[rotate=-135]
\Vertices[unit=2]{circle}{A,B,C,E}
\end{scope}
\NOEA[unit=1.414](E){D}
\Edges(A,B,E,D,C,E,A,C,B)
\end{tikzpicture}
\end{figure}


\begin{figure}[!h]
	\centering
\begin{tikzpicture}
\SetGraphUnit{3}
\GraphInit[vstyle=Shade]
\tikzset{LabelStyle/.style= {draw,
		fill = yellow,
		text = red}}
\Vertex{A}
\EA(A){B}
\EA(B){C}
\SetGraphUnit{6}
% modifie la distance entre les nodes
\NO(B){D}
\Edge[label=1](B)(D)
\tikzset{EdgeStyle/.append style = {bend left}}
\Edge[label=4](A)(B)
\Edge[label=5](B)(A)
\Edge[label=6](B)(C)
\Edge[label=7](C)(B)
\Edge[label=2](A)(D)
\Edge[label=3](D)(C)
\end{tikzpicture}
\caption{La antigua ciudad de Königsberg con tkz-graph}
\end{figure}


\begin{figure}[!h]
	\centering
\begin{tikzpicture}
\draw[help lines] (0,0) grid (3,2);
\pgftransformxshift{1cm}
\pgfpathqmoveto{0pt}{0pt} % not transformed
\pgfpathqlineto{1cm}{1cm} % not transformed
\pgfpathlineto{\pgfpoint{2cm}{0cm}}
\pgfusepath{stroke}
\end{tikzpicture}
\caption{Generando gráficos sobre una malla}
\end{figure}



\begin{figure}[!h]
	\centering
\begin{pgfpicture}
	\pgfpathmoveto{\pgfpointorigin}
	\pgfsetlinetofirstplotpoint
	\pgfplothandlerlineto
	\pgfplotstreamstart
	\pgfplotstreampoint{\pgfpoint{1cm}{0cm}}
	\pgfplotstreampoint{\pgfpoint{2cm}{1cm}}
	\pgfplotstreampoint{\pgfpoint{3cm}{2cm}}
	\pgfplotstreampoint{\pgfpoint{1cm}{2cm}}
	\pgfplotstreamend
	\pgfusepath{stroke}
\end{pgfpicture}

\end{figure}


\begin{figure}[!h]
		\centering
\begin{tikzpicture}
\draw[help lines] (0,0) grid (3,2);
\pgfpathqmoveto{0pt}{0pt}
\pgfpathqcurveto{1cm}{1cm}{2cm}{1cm}{3cm}{0cm}
\pgfusepath{stroke}
\end{tikzpicture}

\end{figure}


\begin{figure}[!h]
		\centering
\begin{pgfpicture}
	\pgfplothandlerrecord{\mystream}
	\pgfplotstreamstart
	\pgfplotstreampoint{\pgfpoint{1cm}{0cm}}
	\pgfplotstreampoint{\pgfpoint{2cm}{1cm}}
	\pgfplotstreampoint{\pgfpoint{3cm}{1cm}}
	\pgfplotstreampoint{\pgfpoint{1cm}{2cm}}
	\pgfplotstreamend
	\pgfplothandlerlineto
	\mystream
	\pgfplothandlerclosedcurve
	\mystream
	\pgfusepath{stroke}
\end{pgfpicture}


\end{figure}


\begin{figure}[!h]
		\centering
\pgfdeclareverticalshading{rainbow}{100bp}
{color(0bp)=(red); color(25bp)=(red); color(35bp)=(yellow);
	color(45bp)=(green); color(55bp)=(cyan); color(65bp)=(blue);
	color(75bp)=(violet); color(100bp)=(violet)}
\begin{tikzpicture}[shading=rainbow]
\shade (0,0) rectangle node[white] {\textsc{pride}} (2,1);
\shade[shading angle=90] (3,0) rectangle +(1,2);
\end{tikzpicture}
\end{figure}


\begin{figure}[!h]
		\centering
\pgfdeclareverticalshading{myshadingF}{100bp}
{color(0bp)=(red); color(25bp)=(green); color(75bp)=(blue); color(100bp)=(black)}
\begin{tikzpicture}
\draw (50bp,50bp) node {\pgfuseshading{myshadingF}};
\draw[white,thick] (25bp,25bp) rectangle (75bp,75bp);
\draw (50bp,0bp) node[below] {first two applications};
\begin{scope}[xshift=5cm]
\draw (50bp,50bp) node{\pgfuseshading{myshadingF}};
\draw[rotate around={45:(50bp,50bp)},white,thick] (25bp,25bp) rectangle (75bp,75bp);
\draw (50bp,0bp) node[below] {third application};
\end{scope}
\begin{scope}[xshift=10cm]
\draw (50bp,50bp) node{\pgfuseshading{myshadingF}};
\draw[white,thick] (50bp,50bp) circle (25bp);
\draw (50bp,0bp) node[below] {fourth application};
\end{scope}
\end{tikzpicture}

\end{figure}


\begin{figure}[!h]
		\centering
\pgfdeclareradialshading{ballshading}{\pgfpoint{-10bp}{10bp}}
{color(0bp)=(red!15!white); color(9bp)=(red!75!white);
	color(18bp)=(red!70!black); color(25bp)=(red!50!black); color(50bp)=(black)}
\pgfuseshading{ballshading}
\hskip 1cm
\begin{pgfpicture}
	\pgfpathrectangle{\pgfpointorigin}{\pgfpoint{1cm}{1cm}}
	\pgfshadepath{ballshading}{0}
	\pgfusepath{}
	\pgfpathcircle{\pgfpoint{3cm}{0cm}}{1cm}
	\pgfshadepath{ballshading}{0}
	\pgfusepath{}
	\pgfpathcircle{\pgfpoint{6cm}{0cm}}{1cm}
	\pgfshadepath{ballshading}{45}
	\pgfusepath{}
\end{pgfpicture}
\end{figure}


\begin{figure}[!h]
		\centering
\begin{tikzpicture}
\def\pgfmatrixbegincode{\node[draw]\bgroup}
\def\pgfmatrixendcode{\egroup;}
\pgfmatrix{rectangle}{center}{mymatrix}
{\pgfusepath{}}{\pgfpointorigin}{\let\&=\pgfmatrixnextcell}
{
	a \& b \& c \\
	d \& \& e \\
}
\end{tikzpicture}
\end{figure}


\begin{figure}[!h]
	\centering
\begin{tikzpicture}
\draw[help lines] (0,0) grid (3,2);
\pgftransformscale{2}
\pgftransformrotate{30}
\pgftransformxshift{1cm}
{\color{red}\pgftext{rotado}}
\pgftransformresetnontranslations
\pgftext{trasladado}
\end{tikzpicture}
\end{figure}



\end{document}


\begin{figure}[!h]
	\centering
	\caption{}
\end{figure}	
	


\begin{figure}[!h]
	\centering
	\begin{tikzpicture}
	\pgfsetfillpattern{stars}{red}
	\filldraw (0,0) rectangle (1.5,2);
	\pgfsetfillpattern{green stars}{red}
	\filldraw (1.5,0) rectangle (3,2);
	\end{tikzpicture}
	\caption{Creando patrones repetitivos}
\end{figure}


