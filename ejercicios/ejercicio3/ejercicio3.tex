% ###################    CONFIGURACIÓN ################### 

% Configuración de estilos generales de la página
\documentclass[12pt,
               % twocolumn
              ]{article}
              
\usepackage[paperheight=297mm,
			paperwidth=210mm,
			tmargin=28mm, 
			headheight=0mm, 
			headsep=0mm,
			textheight=240mm,
			footskip=7mm,
			textwidth=159.2mm,
			bindingoffset=10mm,
			twoside
			]{geometry} 
% -------------------------------------------------------   


% Configuración de párrafos
% -------------------------------------------------------         
\setlength{\parindent}{0em} % Tabulación de un párrafo nuevo
\setlength{\parskip}{1em}   % Distancia en blanco entre un párrafo y otro.
\setlength{\columnsep}{1em} % Margen izquierdo de columnas
\renewcommand{\baselinestretch}{1.5} % Interlineado. Espacio entre dos renglones.
% ------------------------------------------------------- 


% Paquetes de idioma
\usepackage[utf8]{inputenc}
\usepackage[spanish]{babel}
% -------------------------------------------------------  


% Paquetes generales
\usepackage{enumitem} % Para configurar enumeraciones
\usepackage[usenames,dvipsnames]{xcolor}    % Para colores en el texto

% Definimos un nuevo color
\definecolor{webred}{rgb}{0.5, 0, 0}   % less intense red

\usepackage[bookmarks=true,
			bookmarksnumbered=false, % true means bookmarks in 
									 % left window are numbered                         
			bookmarksopen=false,     % true means only level 1
									 % are displayed.
			colorlinks=true,
			linkcolor=webred,
			%linkcolor=OliveGreen,
			urlcolor=cyan]{hyperref} % Para hiperenlaces

% Definimos una nueva orden para pintar una línea cuando escribamos \HRule
\newcommand{\HRule}{\rule{\linewidth}{1mm}}

% Definimos el título y autor de la página

\title
{
	\HRule
	\begin{flushright}
		\Huge
		\textbf{Ejercicio de enumeración}\\[3mm]
		\Large
		\textbf{con}\\
		\Huge
		\texttt{enumitem}
	\end{flushright}
	\HRule 
}

\author{\large Jonathan Martín Valera}


% ###################   FIN CONFIGURACIÓN ################### 

% ###################	INICIO DEL DOCUMENTO ###################

\begin{document}

	\maketitle
	
	\tableofcontents % Hace el índice de contenidos.
	
	\setcounter{section}{-1}
	
	\section{Introducción}
	
	Contiene una invitación a visitar la capital de Noruega,\href{https://www.visitoslo.com/es/}{Oslo}

	\section{Detallar con \texttt{itemize} }
	
	Qué visitar en Oslo:
	
	\begin{itemize}
		\item \href{https://is.gd/SrdzCR}{Palacio Real} de Oslo
		\item Fortaleza de \href{https://is.gd/IKUxmO}{Akershus}.
		\item Palacio de la Ópera.
		\item Crucero por el fiordo.
		\item Península de \href{https://is.gd/HABCLE}{Bygdøy}.
		\item Barrio \href{https://is.gd/Wro3WO}{Grünerløkka}.
	\end{itemize}

	En la Península de Bygdøy tiene los siguientes museos y actividades:
	
	\begin{itemize}
		\item Museo Folklórico Noruego
		\item Museo Marítimos:
		\begin{itemize}
			 \item Museo Fram
			 \begin{itemize}
				\item Inspección de la exposición ambientada dentro de la nave polar Fram
				(adelante), construida en 1892.
				\item Expedición de Fridtjof Nansen (1893-1896).
				\item Expedición de Otto Sverdrup (1898-1902).
				\item Expedición de Roald Amundsen (1910-1912)
			\end{itemize}
	 	\item Museo de los Barcos Vikingos
		\item Museo de Kon-Tiki
		\item Museo Marítimo Noruego
	\end{itemize}
	\item Oscarshal
	\item Huk \& Paradisbukta beach
	\item Paseo ciclista alrededor de Bygdøy
	\item Bygdøy Royal Manor
	\end{itemize}

	\section{Enumeraciones}
	
	Además de lo detallado en la sección 1 En Oslo puede girar visita a Grünerløkka para:
	
	\begin{enumerate}
		\item Ir de compras
		\begin{enumerate}
			\item Mathallen Food Hall
			\item Inventarium
			\item Schous Bøker
			\item Svovel
			\item Sykkelpikene Oslo
			\item Fosfor lamps and coffee
			\item Probat t-shirts
		\end{enumerate}
	\end{enumerate}

	aunque también puede:
	\begin{enumerate}[resume]
		\item Realizar actividades
		\item Disfrutar de la vida nocturna
		\item Recorrer el río Akerselva	
	\end{enumerate}

\end{document}

% ###################	FIN DEL DOCUMENTO ###################