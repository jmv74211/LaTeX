\documentclass[12pt,
               twocolumn
              ]{article}

% a4large.sty -- fill an A4 (210mm x 297mm) page
% Note: 1 inch = 25.4 mm = 72.27 pt
%       1 pt   =  3.5 mm (approx)

% vertical page layout -- one inch margin top and bottom
\topmargin        0   mm    % top margin less 1 inch
\headheight       0   mm    % height of box containing the head
\headsep          0   mm    % space between the head and the body of the page
\textheight     240   mm    % the height of text on the page
\footskip         7   mm    % distance from bottom of body to bottom of foot

% horizontal page layout -- one inch margin each side
\oddsidemargin    0   mm    % inner margin less one inch on odd pages
\evensidemargin   0   mm    % inner margin less one inch on even pages
\textwidth      159.2 mm    % normal width of text on page

\usepackage[utf8]{inputenc}
\usepackage[spanish]{babel}

\usepackage[math]{benton}
\usepackage[T1]{fontenc}

% \usepackage{inconsolata}
% \renewcommand*\familydefault{\ttdefault} 
%                  %% Only if the base font of the document is to be typewriter style
% \usepackage[T1]{fontenc}

\begin{document}

\tableofcontents % Hace el índice de contenidos.

%\section*{Introducción}
\section{Introducción}


\begin{flushleft}
Hoy es \today \newline

\textsl{fiarse}

\end{flushleft}


\begin{flushleft}
  Muestra de documento con contenido tomado del texto de ``El
  Quijote''
\end{flushleft}

\section{Que trata de la condición y ejercicio del famoso hidalgo
  D. Quijote de la Mancha}

En un lugar de la Mancha, de cuyo nombre no quiero acordarme, no ha
mucho tiempo que vivía un hidalgo de los de lanza en astillero, adarga
antigua, rocín flaco y galgo corredor. Una olla de algo más vaca que
carnero, salpicón las más noches, duelos y quebrantos los sábados,
lentejas los viernes, algún palomino de añadidura los domingos,
consumían las tres partes de su hacienda. El resto della concluían
sayo de velarte, calzas de velludo para las fiestas con sus pantuflos
de lo mismo, los días de entre semana se honraba con su vellori de lo
más fino. Tenía en su casa una ama que pasaba de los cuarenta, y una
sobrina que no llegaba a los veinte, y un mozo de campo y plaza, que
así ensillaba el rocín como tomaba la podadera. Frisaba la edad de
nuestro hidalgo con los cincuenta años, era de complexión recia, seco
de carnes, enjuto de rostro; gran madrugador y amigo de la
caza. Quieren decir que tenía el sobrenombre de Quijada o Quesada (que
en esto hay alguna diferencia en los autores que deste caso escriben),
aunque por conjeturas verosímiles se deja entender que se llama
Quijana; pero esto importa poco a nuestro cuento; basta que en la
narración dél no se salga un punto de la verdad.

Es, pues, de saber, que este sobredicho hidalgo, los ratos que estaba
ocioso (que eran los más del año) se daba a leer libros de caballerías
con tanta afición y gusto, que olvidó casi de todo punto el ejercicio
de la caza, y aun la administración de su hacienda; y llegó a tanto su
curiosidad y desatino en esto, que vendió muchas hanegas de tierra de
sembradura, para comprar libros de caballerías en que leer; y así
llevó a su casa todos cuantos pudo haber dellos; y de todos ningunos
le parecían tan bien como los que compuso el famoso Feliciano de
Silva: porque la claridad de su prosa, y aquellas intrincadas razones
suyas, le parecían de perlas; y más cuando llegaba a leer aquellos
requiebros y cartas de desafío, donde en muchas partes hallaba
escrito: la razón de la sinrazón que a mi razón se hace, de tal manera
mi razón enflaquece, que con razón me quejo de la vuestra fermosura, y
también cuando leía: los altos cielos que de vuestra divinidad
divinamente con las estrellas se fortifican, y os hacen merecedora del
merecimiento que merece la vuestra grandeza. Con estas y semejantes
razones perdía el pobre caballero el juicio, y desvelábase por
entenderlas, y desentrañarles el sentido, que no se lo sacara, ni las
entendiera el mismo Aristóteles, si resucitara para sólo ello. No
estaba muy bien con las heridas que don Belianis daba y recibía,
porque se imaginaba que por grandes maestros que le hubiesen curado,
no dejaría de tener el rostro y todo el cuerpo lleno de cicatrices y
señales; pero con todo alababa en su autor aquel acabar su libro con
la promesa de aquella inacabable aventura, y muchas veces le vino
deseo de tomar la pluma, y darle fin al pie de la letra como allí se
promete; y sin duda alguna lo hiciera, y aun saliera con ello, si
otros mayores y continuos pensamientos no se lo estorbaran.

Tuvo muchas veces competencia con el cura de su lugar (que era hombre
docto graduado en Sigüenza), sobre cuál había sido mejor caballero,
Palmerín de Inglaterra o Amadís de Gaula; mas maese Nicolás, barbero
del mismo pueblo, decía que ninguno llegaba al caballero del Febo, y
que si alguno se le podía comparar, era don Galaor, hermano de Amadís
de Gaula, porque tenía muy acomodada condición para todo; que no era
caballero melindroso, ni tan llorón como su hermano, y que en lo de la
valentía no le iba en zaga.

En resolución, él se enfrascó tanto en su lectura, que se le pasaban
las noches leyendo de claro en claro, y los días de turbio en turbio,
y así, del poco dormir y del mucho leer, se le secó el cerebro, de
manera que vino a perder el juicio. Llenósele la fantasía de todo
aquello que leía en los libros, así de encantamientos, como de
pendencias, batallas, desafíos, heridas, requiebros, amores, tormentas
y disparates imposibles, y asentósele de tal modo en la imaginación
que era verdad toda aquella máquina de aquellas soñadas invenciones
que leía, que para él no había otra historia más cierta en el mundo.

Decía él, que el Cid Ruy Díaz había sido muy buen caballero; pero que
no tenía que ver con el caballero de la ardiente espada, que de sólo
un revés había partido por medio dos fieros y descomunales
gigantes. Mejor estaba con Bernardo del Carpio, porque en Roncesvalle
había muerto a Roldán el encantado, valiéndose de la industria de
Hércules, cuando ahogó a Anteo, el hijo de la Tierra, entre los
brazos. Decía mucho bien del gigante Morgante, porque con ser de
aquella generación gigantesca, que todos son soberbios y descomedidos,
él solo era afable y bien criado; pero sobre todos estaba bien con
Reinaldos de Montalbán, y más cuando le veía salir de su castillo y
robar cuantos topaba, y cuando en Allende robó aquel ídolo de Mahoma,
que era todo de oro, según dice su historia. Diera él, por dar una
mano de coces al traidor de Galalón, al ama que tenía y aun a su
sobrina de añadidura.

En efecto, rematado ya su juicio, vino a dar en el más extraño
pensamiento que jamás dio loco en el mundo, y fue que le pareció
convenible y necesario, así para el aumento de su honra, como para el
servicio de su república, hacerse caballero andante, e irse por todo
el mundo con sus armas y caballo a buscar las aventuras, y a
ejercitarse en todo aquello que él había leído, que los caballeros
andantes se ejercitaban, deshaciendo todo género de agravio, y
poniéndose en ocasiones y peligros, donde acabándolos, cobrase eterno
nombre y fama.

Imaginábase el pobre ya coronado por el valor de su brazo por lo menos
del imperio de Trapisonda: y así con estos tan agradables
pensamientos, llevado del estraño gusto que en ellos sentía, se dió
priesa a poner en efecto lo que deseaba. Y lo primero que hizo, fue
limpiar unas armas, que habían sido de sus bisabuelos, que, tomadas de
orín y llenas de moho, luengos siglos había que estaban puestas y
olvidadas en un rincón. Limpiólas y aderezólas lo mejor que pudo; pero
vió que tenían una gran falta, y era que no tenía celada de encaje,
sino morrión simple; mas a esto suplió su industria, porque de
cartones hizo un modo de media celada, que encajada con el morrión,
hacía una apariencia de celada entera. Es verdad que para probar si
era fuerte, y podía estar al riesgo de una cuchillada, sacó su espada,
y le dió dos golpes, y con el primero y en un punto deshizo lo que
había hecho en una semana: y no dejó de parecerle mal la facilidad con
que la había hecho pedazos, y por asegurarse de este peligro, lo tornó
a hacer de nuevo, poniéndole unas barras de hierro por de dentro de
tal manera, que él quedó satisfecho de su fortaleza; y, sin querer
hacer nueva experiencia de ella, la diputó y tuvo por celada finísima
de encaje. Fue luego a ver a su rocín, y aunque tenía más cuartos que
un real, y más tachas que el caballo de Gonela, que tantum pellis, et
ossa fuit, le pareció que ni el Bucéfalo de Alejandro, ni Babieca el
del Cid con él se igualaban. Cuatro días se le pasaron en imaginar qué
nombre le podría: porque, según se decía él a sí mismo, no era razón
que caballo de caballero tan famoso, y tan bueno él por sí, estuviese
sin nombre conocido; y así procuraba acomodársele, de manera que
declarase quien había sido, antes que fuese de caballero andante, y lo
que era entones: pues estaba muy puesto en razón, que mudando su señor
estado, mudase él también el nombre; y le cobrase famoso y de
estruendo, como convenía a la nueva orden y al nuevo ejercicio que ya
profesaba: y así después de muchos nombres que formó, borró y quitó,
añadió, deshizo y tornó a hacer en su memoria e imaginación, al fin le
vino a llamar ROCINANTE, nombre a su parecer alto, sonoro y
significativo de lo que había sido cuando fue rocín, antes de lo que
ahora era, que era antes y primero de todos los rocines del
mundo. Puesto nombre y tan a su gusto a su caballo, quiso ponérsele a
sí mismo, y en este pensamiento, duró otros ocho días, y al cabo se
vino a llamar DON QUIJOTE, de donde como queda dicho, tomaron ocasión
los autores de esta tan verdadera historia, que sin duda se debía
llamar Quijada, y no Quesada como otros quisieron decir. Pero
acordándose que el valeroso Amadís, no sólo se había contentado con
llamarse Amadís a secas, sino que añadió el nombre de su reino y
patria, por hacerla famosa, y se llamó Amadís de Gaula, así quiso,
como buen caballero, añadir al suyo el nombre de la suya, y llamarse
DON QUIJOTE DE LA MANCHA, con que a su parecer declaraba muy al vivo
su linaje y patria, y la honraba con tomar el sobrenombre della.

Limpias, pues, sus armas, hecho del morrión celada, puesto nombre a su
rocín, y confirmándose a sí mismo, se dió a entender que no le faltaba
otra cosa, sino buscar una dama de quien enamorarse, porque el
caballero andante sin amores, era árbol sin hojas y sin fruto, y
cuerpo sin alma. Decíase él: si yo por malos de mis pecados, por por
mi buena suerte, me encuentro por ahí con algún gigante, como de
ordinario les acontece a los caballeros andantes, y le derribo de un
encuentro, o le parto por mitad del cuerpo, o finalmente, le venzo y
le rindo, ¿no será bien tener a quién enviarle presentado, y que entre
y se hinque de rodillas ante mi dulce señora, y diga con voz humilde y
rendida: yo señora, soy el gigante Caraculiambro, señor de la ínsula
Malindrania, a quien venció en singular batalla el jamás como se debe
alabado caballero D. Quijote de la Mancha, el cual me mandó que me
presentase ante la vuestra merced, para que la vuestra grandeza
disponga de mí a su talante? ¡Oh, cómo se holgó nuestro buen
caballero, cuando hubo hecho este discurso, y más cuando halló a quién
dar nombre de su dama! Y fue, a lo que se cree, que en un lugar cerca
del suyo había una moza labradora de muy buen parecer, de quien él un
tiempo anduvo enamorado, aunque según se entiende, ella jamás lo supo
ni se dió cata de ello. Llamábase Aldonza Lorenzo, y a esta le pareció
ser bien darle título de señora de sus pensamientos; y buscándole
nombre que no desdijese mucho del suyo, y que tirase y se encaminase
al de princesa y gran señora, vino a llamarla DULCINEA DEL TOBOSO,
porque era natural del Toboso, nombre a su parecer músico y peregrino
y significativo, como todos los demás que a él y a sus cosas había
puesto.

\section{Que trata de la primera salida que de su tierra hizo el ingenioso D. Quijote}
\label{segundo_capitulo}


\begin{flushleft}
Hechas, pues, estas prevenciones, no quiso aguardar más tiempo a poner
en efecto su pensamiento, apretándole a ello la falta que él pensaba
que hacía en el mundo su tardanza, según eran los agravios que pensaba
deshacer, tuertos que enderezar, sinrazones que enmendar, y abusos que
mejorar, y deudas que satisfacer; y así, sin dar parte a persona
alguna de su intención, y sin que nadie le viese, una mañana, antes
del día (que era uno de los calurosos del mes de Julio), se armó de
todas sus armas, subió sobre Rocinante, puesta su mal compuesta
celada, embrazó su adarga, tomó su lanza, y por la puerta falsa de un
corral, salió al campo con grandísimo contento y alborozo de ver con
cuánta facilidad había dado principio a su buen deseo. Mas apenas se
vió en el campo, cuando le asaltó un pensamiento terrible, y tal, que
por poco le hiciera dejar la comenzada empresa: y fue que le vino a la
memoria que no era armado caballero, y que, conforme a la ley de
caballería, ni podía ni debía tomar armas con ningún caballero; y
puesto qeu lo fuera, había de llevar armas blancas, como novel
caballero, sin empresa en el escudo, hasta que por su esfuerzo la
ganase.  
\end{flushleft}

\begin{flushleft}
Estos pensamientos le hicieron titubear en su propósito; mas pudiendo
más su locura que otra razón alguna, propuso de hacerse armar
caballero del primero que topase, a imitación de otros muchos que así
lo hicieron, según él había leído en los libros que tal le tenían. En
lo de las armas blancas pensaba limpiarlas de manera, en teniendo
lugar, que lo fuesen más que un armiño: y con esto se quietó y
prosiguió su camino, sin llevar otro que el que su caballo quería,
creyendo que en aquello consistía la fuerza de las aventuras. Yendo,
pues, caminando nuestro flamante aventurero, iba hablando consigo
mismo, y diciendo: ¿Quién duda sino que en los venideros tiempos,
ciando salga a luz la verdadera historia de mis famosos hechos, que el
sabio que los escribiere, no ponga, cuando llegue a contar esta mi
primera salida tan de mañana, de esta manera? "Apenas había el
rubicundo Apolo tendido por la faz de la ancha y espaciosa tierra las
doradas hebras de sus hermosos cabellos, y apenas los pequeños y
pintados pajarillos con sus arpadas lenguas habían saludado con dulce
y meliflua armonía la venida de la rosada aurora que dejando la blanda
cama del celoso marido, por las puertas y balcones del manchego
horizonte a los mortales se mostraba, cuando el famoso caballero
D. Quijote de la Mancha, dejando las ociosas plumas, subió sobre su
famoso caballo Rocinante, y comenzó a caminar por el antiguo y
conocido campo de Montiel." (Y era la verdad que por él caminaba) y
añadió diciendo: "dichosa edad, y siglo dichoso aquel adonde saldrán a
luz las famosas hazañas mías, dignas de entallarse en bronce,
esculpirse en mármoles y esculpirse en mármoles y pintarse en tablas
para memoria en lo futuro. ¡Oh tú, sabio encantador, quienquiera que
seas, a quien ha de tocar el ser coronista de esta peregrina historia!
Ruégote que no te olvides de mi buen Rocinante compañero eterno mío en
todos mis caminos y carreras." Luego volvía diciendo, como si
verdaderamente fuera enamorado: "¡Oh, princesa Dulcinea, señora de
este cautivo corazón! Mucho agravio me habedes fecho en despedirme y
reprocharme con el riguroso afincamiento de mandarme no parecer ante
la vuestra fermosura. Plégaos, señora, de membraros de este vuestro
sujeto corazón, que tantas cuitas por vuestro amor padece."  
\end{flushleft}

\end{document}