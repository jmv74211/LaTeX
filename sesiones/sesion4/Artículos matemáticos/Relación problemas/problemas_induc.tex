\section{Inducción}

\subsection{Razonamiento escueto}

\begin{exercise}
  \label{ex:monotonia}
  Demuestre que para todo número natural $n$, $n!n<(n+1)!(n+1)$.
\end{exercise}

\begin{solution}
  Para todo número natural $n$:
  \begin{equation*}
    0<n^{2}+3n+1
  \end{equation*}
  por lo que:
  \begin{equation*}
    n<n^{2}+2n+1
  \end{equation*}
  y por tanto:
  \begin{equation*}
    n!n<n!(n^{2}+2n+1)
  \end{equation*}
  Se tiene las siguiente igualdades:
  \begin{align*}
    n!n&<n!(n(n+2)+1)\\
       &=n!(n(n+1)+n+1)\\
       &=n!n(n+1)+n!(n+1)\\
       &=(n+1)!n+(n+1)!\\
       &=(n+1)!(n+1)
  \end{align*}
  % \begin{align*}
  %   (n+1)!(n+1)&=(n+1)!n+(n+1)!\\
  %              &=n!n(n+1)+n!(n+1)\\
  %              &=n!(n(n+1)+n+1)\\
  %              &=n!(n(n+2)+1)\\
  %              &>n!n
  % \end{align*}
\end{solution}

\begin{exercise}
  Demuestre que para todo número natural $n$ tal que $3\leq n$ se
  cumple:
  \begin{equation*}
    6<n!n
  \end{equation*}
\end{exercise}

\begin{solution}
  Por inducción sobre $n$. Si $n=3$, $3!3=18$ y $6<18$. Supongamos que
  $3\leq n$ y que $6<n!n$. Entonces:
  \begin{align*}
    6&<n!n\\
     &<(n+1)!(n+1)
  \end{align*}
  Por tanto la propiedad es cierta.
\end{solution}

\begin{exercise}
  Demuestre que para todo número natural superior a $4$, $6(n+1)<n!$.
\end{exercise}

\begin{solution}
  Por inducción sobre $n$. Para $n=5$, $6(5+1)=36<120=5!$. Supongamos
  que $5\leq n$ y $6(n+1)<n!$ y demostremos que $6(n+2)<(n+1)!$. Se tiene:
  \begin{align*}
    (n+1)!-n!-6&=n!(n+1)-n!-6\\
               &=n!n-6\\
               &>0
  \end{align*}
  luego $n!+6<(n+1)!$; pero
  \begin{align*}
    6(n+2)&=6(n+1)+6\\
          &<n!+6\\
          &<n!+n!n\\
          &=n!(n+1)\\
          &=(n+1)!
  \end{align*}
\end{solution}

\begin{exercise}
  Demuestre que para todo número natural no nulo se cumple
  \begin{equation*}
    n!<n!(n^{2}+n+1)
  \end{equation*}
\end{exercise}

\begin{solution}
  Es evidente a partir del hecho de que para todo número natural $n$,
  $n!$ es no negativo y de que para todo número natural no nulo $n$,
  $1<n^{2}+n+1$.
\end{solution}

\begin{exercise}
  Demuestre que para todo número natural $n$ tal que $5\leq n$,
  \begin{equation*}
    6(n+1)<n!(n^{2}+n+1)
  \end{equation*}
\end{exercise}

\begin{solution}
  Para todo número natural $n$ tal que $n\leq 5$, $6(n+1)<n!$ y
  $n!<n!(n^{2}+n+1)$; por tanto, $6(n+1)<n!(n^{2}+n+1)$.
\end{solution}

\begin{exercise}
  Demuestre que para todo número natural $n$ tal que $3\leq n$,
  \begin{equation*}
    6(n+1)<n!(n^{2}+n+1)
  \end{equation*}
\end{exercise}

\begin{solution}
  Conocemos la veracidad de la relación para todo número natural $n$
  tal que $5\leq n$. Basta comprobar que también vale para $n=3$ y
  $n=4$.
\end{solution}

\begin{exercise}
  Demuestre que para todo número natural $n$ tal que $3\leq n$,
  \begin{equation*}
    6(n+1)+n!n<(n+1)!(n+1)
  \end{equation*}
\end{exercise}

% \begin{solution}
%   Por inducción sobre $n$. Supongamos que $n=3$,
%   $3!3+6(3+1)=42<96=(3+1)!(3+1)$. Supongamos que $3\leq n$ y que
%   $6(n+1)+n!n<(n+1)!(n+1)$; demostremos que
% \end{solution}

\begin{solution}
  Consideremos las siguientes igualdades:
  \begin{align*}
    (n+1)!(n+1)-n!n-6(n+1)&=(n+1)!n+(n+1)!-n!n-6(n+1)\\
                          &=(n+1)!n+n!n+n!-n!n-6(n+1)\\
                          &=(n+1)!n+n!-6(n+1)\\
                          &=n!(n+1)n+n!-6(n+1)\\
                          &=n!(n^{2}+n+1)-6(n+1)
  \end{align*}
  luego lo que queremos demostrar es equivalente a que
  $6(n+1)<n!(n^{2}+n+1)$ y sabemos que esto es cierto para todo número
  natural $n$ tal que $3\leq n$.
\end{solution}

\begin{exercise}
  Para todo número natural $n$ son equivalentes las siguientes
  desigualdades:
  \begin{enumerate}
  \item $n!+3n^{2}+3n+1<(n+1)!$
  \item $3n(n+1)+1<n!n$
  \end{enumerate}
\end{exercise}

\begin{solution}
  Basta considerar lo siguiente:
  \begin{align*}
    n!+3n^{2}+3n+1<(n+1)!&\text{ sii } 0<(n+1)!-n!-3n^{2}-3n-1\\
                         &\text{ sii } 0<n!(n+1)-n!-3n^{2}-3n-1\\
                         &\text{ sii } 0<n!n-3n(n+1)-1\\
                           &\text{ sii } 3n(n+1)+1<n!n
  \end{align*}
\end{solution}

\begin{exercise}
  Demuestre que para todo número natural $n$ tal que $4\leq n$,
  $3n(n+1)-1<n!n$.
\end{exercise}

\begin{solution}
  Por inducción sobre $n$. Supongamos $n=4$;
  \begin{equation*}
    3\cdot 4\cdot 5-1=59<96=4\cdot3\cdot2\cdot4=4!4
  \end{equation*}
  Supongamos ahora que $n$ es un número natural cumpliendo $4\leq n$ y
  que $3n(n+1)-1<n!n$. Se tiene:
  \begin{align*}
    3(n+1)(n+2)-1&=3(n+1)n+6(n+1)-1\\
                 &<n!n+6(n+1)\\
                 &<(n+1)!(n+1)
  \end{align*}
\end{solution}

\begin{exercise}
  Demuestre que para todo número natural $n$ tal que $4\leq n$,
  \begin{equation*}
    n!+3n^{2}+3n+1<(n+1)!
  \end{equation*}
\end{exercise}

\begin{solution}
  Por sí mismo, como conclusión inmediata de lo anterior.
\end{solution}

\begin{exercise}
  Demuestre que para todo número natural $n$ tal que $6\leq n$ se cumple $n^{3}<n!$
\end{exercise}

\begin{solution}
  Por inducción sobre $n$. Supongamos $n=6$:
  \begin{equation*}
    6^{3}=216<720=6!
  \end{equation*}
  Supongamos que $6\leq n$ y que $n^{3}<n!$. Entonces:
  \begin{align*}
    (n+1)^{3}&=n^{3}+3n^{2}+3n+1\\
             &<n!+3n^{2}+3n+1\\
             &<(n+1)!
  \end{align*}
\end{solution}

\subsection{Razonamiento más profundo}

\begin{definition}
  \label{df:parDivisor}
  Sean $a,b\in \mathbb{Z}$ tales que $b\neq 0$. Un \textit{par divisor
    de} $a$ \textit{por} $b$ \index{par divisor} es un par de enteros
  $\langle q,r\rangle$ tales que:
  \begin{enumerate}
  \item $a=bq+r$.
  \item $0\leq r<|b|$.
  \end{enumerate}
\end{definition}

\begin{lemma}
  \label{lm:casoNegativos}
  Supongamos que para todo $a,b\in\mathbb{N}$ tales que $b\neq 0$
  existe un par divisor de $a$ por $b$. Entonces para todo
  $a,b\in\mathbb{Z}$ tales que $b\neq 0$ existe un par divisor de $a$
  por $b$ y es el siguiente por casos, siempre que
  $\langle q,r\rangle$ sea un par divisor de $|a|$ por $|b|$:
  \begin{enumerate}
  \item Si $0<a$ y $b<0$,
    $\langle\operatorname{sgn}(a)\operatorname{sgn}(b)q,r\rangle$ es
    un par divisor de $a$ por $b$.
  \item En el resto de casos, si $r=0$ entoces
    $\langle\operatorname{sgn}(a)\operatorname{sgn}(b)(q+\operatorname{sgn}(r)),0\rangle$
    es un par divisor de $a$ por $b$ y si $r\neq 0$, entonces $\langle\operatorname{sgn}(a)\operatorname{sgn}(b)(q+\operatorname{sgn}(r)),|b|-r\rangle$
    es un par divisor de $a$ por $b$.
  \end{enumerate}
\end{lemma}

\begin{proof}
  Supongamos que $\langle q,r\rangle$ es un par divisor de $|a|$ por
  $|b|$. La casuística es la siguiente:
  \begin{enumerate}
  \item $0<a$ y $b<0$; entonces 
    \begin{align*}
      a&=(-b)q+r\\
       &=b(-q)+r
    \end{align*}
    y $0\leq r<|b|$. Así pues,
    $\langle
    -q,r\rangle=\langle\operatorname{sgn}(a)\operatorname{sgn}(b)q,r\rangle$.
  \item $a<0$ y $b<0$; $-a=(-b)q+r$ y $0\leq r<-b$. Entonces:
    \begin{itemize}
    \item $r=0$; como $-a=(-b)q$ se cumple $a=bq$ y así
      $\langle q,0\rangle =
      \langle\operatorname{sgn}(a)\operatorname{sgn}(b)(q+\operatorname{sgn}(r)),0\rangle$
      es un par divisor de $a$ por $b$.
    \item $r>0$; entonces:
      \begin{align*}
        a&=bq-r\\
         &=bq+0-r\\
         &=bq+b-b-r\\
         &=b(q+1)+(-b-r)
      \end{align*}
      Las condiciones $0\leq r<-b$ son equivalentes a $b<-r\leq 0$ o
      también a $0<-b-r<|b|$. En definitiva, se tiene que
      $\langle q+1,|b|-r\rangle
      =\langle\operatorname{sgn}(a)\operatorname{sgn}(b)(q+\operatorname{sgn}(r)),|b|-r\rangle$
      es un par divisor de $a$ por $b$.
    \end{itemize}
  \item Si $a<0$ y $0<b$; en este caso $-a=qb+r$ y $0\leq r<b$.
    \begin{itemize}
    \item $r=0$; como $-a=bq$ se cumple $a=b(-q)$ y así
      $\langle -q,0\rangle =
      \langle\operatorname{sgn}(a)\operatorname{sgn}(b)(q+\operatorname{sgn}(r)),0\rangle$
      es un par divisor de $a$ por $b$.
    \item $0<r$; entonces:
      \begin{align*}
        a&=-bq-r\\
         &=b(-q)-r\\
         &=b(-q)+0-r)\\
         &=b(-q)-b+b-r\\
         &=b(-q-1)+(b-r)\\
         &=b(-q-1)+(|b|-r)\\
      \end{align*}
      y también $0<b-r<b$. En definitiva, $\langle -q-1,b-r\rangle =
      \langle\operatorname{sgn}(a)\operatorname{sgn}(b)(q+\operatorname{sgn}(r)),|b|-r\rangle$
    \end{itemize}
  \end{enumerate}
\end{proof}

\begin{theorem}[teorema de la división]
  \label{th:teoremaDeLaDivision} \index{teorema ! de la división} Para
  todo $a,b\in\mathbb{Z}$ tales que $b\neq 0$ existe un único par
  divisor de $a$ por $b$.
\end{theorem}

\begin{proof}
  Supongamos que $a,b\in\mathbb{N}$ y que $b\neq 0$. La demostraci\'on
  de la \textbf{existencia} es por inducci\'on. Si $a=0$, en tal caso
  $0=b0+0$ y así $\langle 0,0\rangle$ es un par divisor de $a$ por
  $b$. Supongamos que $a\neq 0$ y que para todo $0\leq c<a$ existen
  $q_{c}$ y $r_{c}$ tales que $c=q_{c}b+r_{c}$. Podemos distinguir dos
  casos:
  \begin{enumerate}
  \item Si $a<b$ entonces $a=0b+a$, como $0\leq a<b$ podemos tomar $q=0$
    y $r=a$.
  \item Si $b\leq a$, sea $c=a-b$. En este caso $0\leq c<a$. Por
    hip\'otesis de inducci\'on, $c=q_{c}b+r_{c}$, para ciertos $q_{c}$ y
    $r_{c}$ tales que $0\leq r_{c}<a$. Pero entonces:
    \begin{align*}
      a&=b+c\\
       &=b+q_{c}b+r_{c}\\
       &=b(q_{c}+1)+r_{c}
    \end{align*}
    y $0\leq r_{c}<b$. As\'{\i} pues, en este caso podemos tomar
    $q=q_{c}+1$ y $r=r_{c}$.
  \end{enumerate}
  Por el principio de inducci\'on la existencia queda probada para
  todo $a,b\in\mathbb{N}$ tal que $b\neq 0$. La existencia en los
  restantes casos está garantizada por lo establecido en el
  \hyperref[lm:casoNegativos]{Lema \ref*{lm:casoNegativos}}. Para la
  \textbf{unidad}, supongamos que existen $q,q',r,r'\in \mathbb{Z}$
  tales que $a=bq+r=bq'+r'$ y que $0\leq r<|b|$ y que $0\leq
  r'<|b|$. Supongamos, para fijar ideas, que $r\leq r'$ y consideremos
  $a-a$, o sea, $0=b(q-q')+(r-r')$, o $b(q-q')=r'-r$. Se pueden dar
  dos casos:
  \begin{enumerate}
  \item $0<b$. Como $0\leq r'-r\leq r'$ y $r'<b$, se tiene $0\leq
    b(q-q')<b$. De donde, $0\leq q-q'<1$ y por tanto $q-q'=0$, o sea,
    $q=q'$. Entonces $r'-r=b(q-q')=0$, de donde $r'=r$.
  \item $b<0$. Como $r'-r=b(q-q')$ y $0\leq r'-r\leq r\leq -b$, entonces
    $0\leq b(q-q')<-b$. Por tanto, $0\leq q'-q<1$. As\'{\i} pues,
    $q'-q=0$, o equivalentemente, $q'=q$. Como antes deducimos que
    $r=r'$.
  \end{enumerate}
\end{proof}

\begin{exercise}
  \label{ex:DeMoivre}
  Demuestre que para todo número entero $n$ y para todo número
  complejo no nulo $z$, que expresado en forma polar sea ---digamos---
  $\cos\theta_{z}+i\sin\theta_{z}$, se cumple que:
  \begin{equation}
    \label{eq:DeMoivre}
    z^{n}=\cos(n\theta_{z})+i\sin(n\theta_{z})
  \end{equation}
  Concluya que para todo número complejo no nulo
  $z=r_{z}(\cos\theta_{z}+i\sin\theta_{z})$ y todo número entero $n$
  se cumple (fórmula de \textit{De Moivre}):\footnote{Recuerde que si
    $z=a+bi$ es cualquier número complejo no nulo, entonces
    $z^{-1}=\frac{a-bi}{a^{2}+b^{2}}$; es decir
    $z^{-1}=\frac{\bar{z}}{|z|^{2}}$ lo que es deducido sin más que
    observar que:
    \begin{equation*}
      \frac{1}{z}=\frac{\bar{z}}{z\bar{z}}
    \end{equation*}
  }
  \begin{equation}
    \label{eq:DeMoivreGen}
    z^{n}=r_{z}^{n}(\cos(n\theta_{z})+i\sin(n\theta_{z}))
  \end{equation}  
\end{exercise}
  %http://matematicatuya.com/Complejos/Inverso-multiplicativo-imaginario.html
  %https://es.wikipedia.org/wiki/F%C3%B3rmula_de_De_Moivre

\begin{solution}
  En primer lugar haremos la demostración de la
  \hyperref[eq:DeMoivre]{igualdad (\ref*{eq:DeMoivre})} cuando $n$ es
  natural y será por medio del principio de inducción finita. En el
  \textbf{caso base}, tenemos por una parte que:
  \begin{equation*}
    \cos(0\theta_{z})+i\sin(0\theta_{z})=1+0i=1=z^{0}
  \end{equation*}
  Supongamos que $n$ es un número natural y que
  $z^{n}=\cos(n\theta_{z})+i\sin(n\theta_{z})$ (\textbf{hipótesis de
    inducción}) y demostremos, en el \textbf{paso de inducción}, que
  $z^{n+1}=\cos((n+1)\theta_{z})+i\sin((n+1)\theta_{z})$. En efecto:
  \begin{align*}
    z^{n+1}&=(\cos\theta_{z}+i\sin\theta_{z})^{n+1}\\
           &=(\cos\theta_{z}+i\sin\theta_{z})^{n}(\cos\theta_{z}+i\sin\theta_{z})\\
           &=(\cos(n\theta_{z})+i\sin(n\theta_{z}))(\cos\theta_{z}+i\sin\theta_{z})\\
    &=\cos(n\theta_{z})
      \cos\theta_{z}-\sin(n\theta_{z})\sin(\theta_{z})
      +i(\cos(n\theta_{z})
      \sin\theta_{z}+\sin(n\theta_{z})\cos(\theta_{z}))\\
    &=\cos((n+1)\theta_{z})+i\sin((n+1)\theta_{z})
  \end{align*}
  Por el Principio de Inducción Finita, la
  \hyperref[eq:DeMoivre]{igualdad (\ref*{eq:DeMoivre})} vale para todo
  número natural $n$. Si $n<0$, sea $m=|n|=-n$. Entonces:
  \begin{align*}
    (\cos\theta_{z}+i\sin\theta_{z})^{n}&=(\cos\theta_{z}+i\sin\theta_{z})^{-m}\\
                                        &=\frac{1}{(\cos\theta_{z}+i\sin\theta_{z})^{m}}\\
                                        &=\frac{1}{\cos(m\theta_{z})+i\sin(m\theta_{z})}\\
                                        &=\cos(m\theta_{z})-i\sin(m\theta_{z})\\
                                        &=\cos(-m\theta_{z})+i\sin(-m\theta_{z})\\
    &=\cos(n\theta_{z})+i\sin(n\theta_{z})
  \end{align*}
  Por lo que la \hyperref[eq:DeMoivre]{igualdad (\ref*{eq:DeMoivre})}
  vale para todo número entero $n$. Supongamos ahora que
  $z=r_{z}(\cos\theta_{z}+i\sin\theta_{z})$. Entonces:
  \begin{align*}
    z^{n}&=(r_{z}(\cos\theta_{z}+i\sin\theta_{z}))^{n}\\
         &=r_{z}^{n}(\cos\theta_{z}+i\sin\theta_{z})^{n}\\
         &=r_{z}^{n}(\cos(n\theta_{z})+i\sin(n\theta_{z}))
  \end{align*}
  lo que demustra la \hyperref[eq:DeMoivreGen]{igualdad (\ref*{eq:DeMoivreGen})}.
\end{solution}

\begin{exercise}
  Pruebe que el producto de tres números naturales consecutivos
  cualesquiera es divisible por $6$.
\end{exercise}

\begin{solution}
  Sea $p$ la aplicación entre números naturales que al número natural
  $n$ cualquiera le asigna $p(n)=n(n+1)(n+2)$. El razonamiento es por
  inducción sobre $n$ según el enunciado $P(i)$ del tenor ``$6$ divide
  a $p(i)$''. Como $p(0)=0$ está claro que $P(0)$ es cierta
  (\textbf{caso base}). Supongamos que $k$ es un número natural y que
  $P(k)$ es cierta (\textbf{hipótesis de inducción}), es decir, que
  existe un número natural $k'$ tal que $p(k)=6k'$; demostremos (en
  el \textbf{paso de inducción}) que como consecuencia $P(k+1)$ es
  cierta. Para ello consideremos:
  \begin{align*}
    p(k+1)-p(k)&=(k+1)(k+2)(k+3)-k(k+1)(k+2)\\
               &=(k+3-k)(k+2)(k+3)\\
               &=3(k+1)(k+2)
  \end{align*}
  Al ser consecutivos los números $k+1$ y $k+2$, exactamente uno de
  los dos es par, por lo que $(k+1)(k+2)$ es par, o sea, existirá un
  número natural $k''$ tal que $(k+1)(k+2)=2k''$. Recapitulando,
  tenemos que:
  \begin{align*}
    6k''&=3(2k'')\\
        &=p(k+1)-p(k)\\ 
        &=p(k+1)-6k'
  \end{align*}
  y así pues $p(k+1)=6k''+6k'=6(k''+k')$, o equivalentemente,
  $6\mid p(k+1)$. Por el principio de inducción finita sabemos que
  para todo número natural $i$, $P(i)$ es cierta y ello es lo que se
  quería demostrar.
\end{solution}

\begin{exercise}
  \label{ex:mcm}
  Cualesquiera dos números naturales $a$ y $b$ tienen un mínimo común
  múltiplo, esto es, un número $m$ que es múltiplo común a $a$ y $b$ y
  es divisor de cualquier otro múltiplo común a
  ambos.\footnote{Cuídese de sustituir la expresión ``es divisor de''
    con la de ``es menor o igual que''. Si así fuera, $15$ no podría
    ser mínimo común múltiplo de $3$ y $5$, ya que no es menor o igual
    que $-15$, cuando $15$ y $(-1)15$ cumplen lo necesario para serlo
    y, por convenio, se ha elegido en el caso de los números enteros
    al positivo entre los dos asociados como ``el'' mínimo común
    múltiplo. Sin salir del ámbito de los número naturales, más
    doloroso sería razonar con $0$, que es múltiplo de cualquier
    entero; nuestro error lo convertiría en el mínimo común múltiplo
    de cualquier pareja de números naturales.}
\end{exercise}
% Childs, 3ª edc. pág. 21. Demostración sugerida que dista mucho de
% ser completa. ¡Parece que Lindsay se ha olvidado del 0!

\begin{solution}
  La \textbf{primera} solución que damos es \textbf{vía} el
  \textbf{Principio del Buen Orden}. Si $a=0$ ó $b=0$, entonces el
  único múltiplo común a ambos es $0$, de hecho, el mínimo común
  múltiplo de $a$ y $b$. Supongamos ahora que ninguno de ellos es nulo
  y llamemos $M_{ab}$ al conjunto de los naturales múltiplos comunes a
  $a$ y $b$ y $V_{ab}=M_{ab}\setminus\{0\}$.  El conjunto $V_{ab}$ es
  no vacío pues al menos contiene a $ab$. Por el \textit{Principio de
    Buen Orden}, deberá contener un elemento mínimo $m$ y éste será
  mínimo común múltiplo de $a$ y de $b$. En efecto:
  \begin{itemize}
  \item $a\mid m$ y $b\mid m$, pues $m\in V_{ab}$.
  \item Supongamos que $n\in V_{ab}$. Por el \textbf{Teorema de la
      División}, existen números naturales $q$ y $r$ tales que
    $n=mq+r$ y $0\leq r<m$. Se tiene que $r\in M_{ab}$, puesto que
    $m,n\in V_{ab}$; pero como $r<m$ y $m$ es el mínimo de $V_{ab}$
    obligatoriamente estará en $M_{ab}\setminus V_{ab}$, es decir,
    $r=0$. Así pues, $m\mid n$.  
  \end{itemize}
  De lo anterior se deduce que $m$ es un mínimo común múltiplo de $a$
  y $b$, de hecho, el único no negativo que es representado por
  $[a,b]$. La \textbf{segunda} solución es \textbf{vía} el concepto de
  máximo común divisor de números enteros $a$ y $b$, esto es, un
  número $m$ que es divisor común a $a$ y $b$ y es múltiplo de
  cualquier otro divisor común a ambos. Si, en un inocente abuso del
  lenguaje, la frase ``$m$ es máximo común divisor de $a$ y $b$'' es
  abreviada por $(a,b)=m$, hemos de destacar dos propiedades:
  \begin{itemize}
  \item $(a,0)=a$ (razone la verdad de esta afirmación)
  \item $(a,b)=(b,a\mod b)$ (demuestre, como sencillo ejercicio
    aritmético, la verdad de esta afirmación)
  \end{itemize}
  Razonemos ahora por el \textbf{Segundo Principio de Inducción
    Finita} según el enunciado $P(k)$ del tenor ``para todo número
  natural $m$, existe $(a,k)$''. Supongamos, como \textbf{hipótesis de
  inducción}, que $n$ es un número natural y que el resultado es cierto
para todo número natural $k$ que cumpla $k<n$. Razonamos por casos:
\begin{itemize}
\item $n=0$; sea cual sea el número natural $m$ se tiene $(m,0)=m$, es
  decir, existe un máximo común divisor de $m$ y $0$ por cumplir $m$
  las propiedades necesarias al efecto.
\item $n\neq 0$; por el Teorema de la División, existen números $q,r$
  tales que $m=nq+r$ y $0\leq r<n$ ($r$ el valor que hemos nombrado
  como $m\mod n$). Como $m\mod n <n$, de la hipótesis de inducción se
  deduce que existe un máximo común divisor de $n$ y $m$ que, según
  sabemos, es un máximo común divisor de $m$ y $n$.
\end{itemize}
Concluimos que $P(k)$ vale para todo número natural $k$. Para
cualesquiera números enteros $m$ y $n$, $(m,n)$ representará al único
entero no negativo que cumple las propiedades de máximo común
divisor. Ahora bien, es fácil entender que:
\begin{itemize}
\item para cualesquiera números naturales $m$ y $n$, $(m,n)\mid m$,
  por lo que existirá un natural $u$ tal que $(m,n)u=mn$.
\item $u$ es un mínimo común múltiplo de $m$ y $n$, por lo que la
  existencia de máximo común divisor es garantía suficiente de la
  existencia de mínimo común múltiplo.
\end{itemize}
\end{solution}

\begin{exercise}[Multiplicación por el \textit{Método del Campesino Ruso}]
  \label{00_induc_16}
  Sea $p$ la función dada por:
  \begin{align*}
    p(a,0)&=0,\\
    p(a,b)&=
            \begin{cases}
              p\left(2a,\frac{b}{2}\right)&\text{ si $b$ es par,}\\
              p\left(2a,\frac{b-1}{2}\right)+a&\text{ si $b$ es impar.}
            \end{cases}
  \end{align*}
  Demuestre por inducción que para cualesquiera números naturales $a$
  y $b$, $p(a,b)=ab$.
\end{exercise}

\begin{solution}
  La demostración es por el \textbf{segundo principio de inducción
    finita} según el enunciado $P(i)$ del tenor:
  \begin{center}
    Para todo número natural $m$, $p(m,i)=mi$
  \end{center}
  Supongamos que $n$ es un número natural y que para todo número
  natural $k<n$ es cierta $P(k)$ (\textbf{hip. de inducción}). En el
  \textbf{paso de inducción} demostraremos que $P(n)$ es cierta. En
  efecto, son posibles dos casos, que distinguiremos por tener
  cada uno un tratamiento distinto:
  \begin{itemize}
  \item $n$ par; en este caso, a su vez, hay dos posibilidades:
    \begin{itemize}
    \item $n=0$; sea cual sea el número natural $m$,
      $p(m,0)=0=m0$. Así pues vale $P(0)$. Obsérvese que en este caso
      no es necesario hacer uso de la hipótesis de inducción.
    \item $n\neq 0$ y $n$ es par; sea cual sea el número natural $m$,
      \begin{align*}
        p(m,n)&=p\left(2m,\frac{n}{2}\right)&&\text{por definición de
                                               }p\\
              &=2m\frac{n}{2}&&\text{de la hip. de inducción, ya que
                                }\frac{n}{2}<n\\
              &=mn&&
      \end{align*}
    \end{itemize}
  \item $n$ es impar; sea cual sea el número natural $m$,
    \begin{align*}
      p(m,n)&=p\left(2m,\frac{n-1}{2}\right)+m&&\text{por definición de
                                             }p\\
            &=2m\frac{n-1}{2}+m&&\text{de la hip. de inducción, ya que
                              }\frac{n-1}{2}<n\\
            &=m(n-1)+m&&\\
            &=mn&&
    \end{align*}
  \end{itemize}
  Así pues, en aplicación del \textit{Segundo Principio de Inducción
    Finita}, $P(i)$ es cierta para todo número natural $i$ y ello es
  lo que se pedía.
\end{solution}

\begin{exercise}
  \label{ex:pvpo2}
  Si $n$ es un número natural cualquiera, sea
  $f(n)=2^{2^{n+1}}+2^{2^{n}}+1$. Demuestre que para todo número
  natural $n$:
  \begin{enumerate}
  \item $(n^{2}-n+1, n^{2}+n+1)=1$
  \item $f(n)$ tiene al menos $n+1$ factores primos distintos.
  \end{enumerate}
\end{exercise}

\begin{solution}~
  % https://www.cut-the-knot.org/proofs/InfinitudeOfPrimesViaPowersOfTwo.shtml
  \begin{enumerate}
  \item Si $p$ fuese un factor primo de $(m^{2}-m+1, m^{2}+m+1)$,
    entonces dividiría a $m^{2}+m+1-(m^{2}-m+1)=2m$; pero como tanto
    $m^{2}-m+1$ como $m^{2}+m+1$ son impares, $p$ debe ser impar. Se
    deduce entonces que $p$ sería un divisor de $m$ y, por tanto, de
    $m^{2}+m$ y $m^{2}+m+1$; como consecuencia el primo $p$ sería un
    divisor de $1$, lo cual es absurdo. Como consecuencia
    $(m^{2}-m+1, m^{2}+m+1)=1$.
  \item El razonamiento es por inducción sobre $n$. El valor de $f(0)$
    es $7$, por lo que $f(0)$ tiene $0+1$, es decir $1$,
    factores. Supongamos, como hipótesis de inducción, que $0\leq n$ y
    que el resultado es cierto para $n$; demostremos que también lo
    será para $n+1$, que será un número natural no nulo. Sea
    $g(x)=x^{4}+x^{2}+1$; si $0<n$ entonces $f(n)=g(2^{2^{n-1}})$ y
    dado que $x^{4}+x^{2}+1=(x^{2}-x+1)(x^{2}+x+1)$ entonces:
      \begin{align*}
        g(2^{2^{n-1}})&=(2^{2^{n}}+2^{2^{n-1}}+1)(2^{2^{n}}-2^{2^{n-1}}+1)&&\\
          &=((2^{2^{n-1}})^{2}+2^{2^{n-1}}+1)((2^{2^{n-1}})^{2}-2^{2^{n-1}}+1)&&\\
          &=(m^{2}-m+1)(m^{2}+m+1)&&(m=2^{2^{n-1}})
      \end{align*}
      Así pues:
      \begin{align*}
        f(n+1)&=2^{2^{n+2}}+2^{2^{n+1}}+1\\
          &=(2^{2^{n+1}}+2^{2^{n}}+1)(2^{2^{n+1}}-2^{2^{n}}+1)\\
          &=f(n)(2^{2^{n+1}}-2^{2^{n}}+1)
      \end{align*}
      Por la hipótesis de inducción, $f(n)$ tiene al menos $n+1$
      factores primos distintos; dado que el número
      $2^{2^{n+1}}-2^{2^{n}}+1$ es mayor o igual que $3$ y
      $(f(n),2^{2^{n+1}}-2^{2^{n}}+1)=1$, entonces
      $2^{2^{n+1}}-2^{2^{n}}+1$ tiene todos sus factores primos ---y
      al menos hay uno--- distintos a los de $f(n)$, es decir,
      $f(n+1)$ tiene al menos $n+2$ factores primos.
    \end{enumerate}
  \end{solution}

  \begin{exercise}
    Demuestre que para todo número natural $n$ vale la siguiente
    igualdad:
    \begin{equation*}
      11\bottominset{\footnotesize 2n)}{\,\ldots}{2.5mm}{}1
      - 22\bottominset{\footnotesize n)}{\,\ldots}{2.5mm}{}2
      = (33\bottominset{\footnotesize n)}{\,\ldots}{2.5mm}{}3)^{2}
    \end{equation*}
  \end{exercise}

  \begin{proof}
    Definamos las siguientes funciones:
    \begin{equation*}
      \begin{matrix*}[l]
      f(0)  = 0          &&g(0)   =0       &&h(0)  =0\\
      f(n+1)=f(n)10^{2}+11&&g(n+1)=g(n)10+2&&h(n+1)=h(n)10+3
    \end{matrix*}      
  \end{equation*}
  En primer lugar demostremos por el \textit{Principio de Inducción
    Finita} que para todo número natural $n$, $3g(n)=2h(n)$; ello
  puede ser sirviéndonos del enunciado $Q(k)$ del tenor:
  \begin{equation}
    \label{eq:arit00ind}
    3g(k)=2h(k)
  \end{equation}
  En efecto, si $n=0$ entonces $3g(0)=3\cdot 0=0=2\cdot 0=2\cdot
  h(0)$, es decir, vale $Q(0)$. Supongamos que para el número natural
  $n$ vale $Q(n)$; entonces:
  \begin{align*}
    3g(n+1)&=3(g(n)10+2)\\
           &=3g(n)10+6\\
           &=2h(n)10+6\\
           &=2(h(n)10+3)\\
           &=2h(n+1)
  \end{align*}
  de lo que deducimos que para todo número natural $k$, vale
  $Q(k)$. Multiplicando ahora la igualdad
  \hyperref[eq:arit00ind]{(\ref*{eq:arit00ind})} por $30$ obtenemos que para
  todo número natural $n$, $9g(n)10=60h(n)$ y por tanto:
  \begin{align*}
    60h(n)&=(10-1)g(n)10\\
          &=g(n)10^{2}-g(n)10
  \end{align*}
  Se tiene, en definitiva, que para todo número natural $n$,
  \begin{equation}
    \label{eq:arit01ind}
    60h(n)-g(n)10^{2}=-g(n)10
  \end{equation}
  Seguidamente demostremos lo que se pide usando el \textit{Principio
    de Inducción Finita} según el enunciado $P(k)$ del tenor:
    \begin{equation*}
      f(k)-g(k)=h(k)^{2}
    \end{equation*}
    En el \textbf{caso base} demostremos que vale $P(0)$. En efecto:
    \begin{equation*}
      f(0)-g(0)=0-0=0=0^{2}=h(0)^{2}
    \end{equation*}
    Supongamos que $n$ es un número natural y que para él es cierto
    $P(n)$ (\textbf{hipótesis de inducción}) y demostremos que en
    consecuencia es cierto $P(n+1)$. En efecto,
    \begin{align*}
      f(n+1)-g(n+1)&=f(n)10^{2}+11-(10g(n)+3)&&\\
                   &=f(n)10^{2}+11-10g(n)-3&&\\
                   &=f(n)10^{2}-g(n)10+9&&\\
                   &=f(n)10^{2}+9+60h(n)-g(n)10^{2}&&
                          \text{por \hyperref[eq:arit01ind]{(\ref*{eq:arit01ind})}}\\
                   &=(f(n)-g(n))10^{2}+9+60h(n)&&\\
                   &=h(n)^{2}10^{2}+3^{2}+2\cdot3h(n)10&&\text{por
                                                               h.i.}\\
                   &=(h(n)10+3)^{2}&&\\
                   &=h(n+1)^{2}&&
    \end{align*}
    lo que significa que para todo número natural $n$, vale
    $P(n)$.\footnote{En internet ha
      \htmladdnormallink{circulado}{https://twitter.com/ilarrosac/status/1045399092619300865}
      la siguiente demostración:
      \begin{align*}
        \frac{10^{2n}-1}{9}-\frac{2(10^{n}-1)}{9}&=\frac{10^{2n}-2\cdot10^{n}+1}{9}\\
                                                &=\frac{(10^{n}-1)^{2}}{3^{2}}\\
                                                &=\left(\frac{10^{n}-1}{3}\right)^{2}.
      \end{align*}
    }
  \end{proof}

  \begin{exercise}
    \label{00_induc_21}
    Los \textit{números de Fibonacci} son los números de la sucesión:
    \begin{equation*}
      f(n)=
      \begin{cases}
        0,&\text{ si }n=0;\\
        1,&\text{ si }n=1;\\
        f(n-1)+f(n-2),& \text{ si }1<n; 
      \end{cases}
    \end{equation*}
    Demuestre que para todo número natural $n$ se cumple:
    \begin{equation*}
      f(n)<\left(\frac{5}{3}\right)^{n}
    \end{equation*}    
  \end{exercise}

  \begin{solution}
    El razonamiento es por el Segundo Principio de Inducción Finita
    según el enunciado $P(k)$ del tenor:
    \begin{equation*}
      f(k)<\left(\frac{5}{3}\right)^{k}
    \end{equation*}
    Supongamos, como hipótesis de inducción, que $n$ es un número
    natural y que para todo número natural $k$ tal que $k<n$, vale $P(k)$. El
    razonamiento es por casos:
    \begin{itemize}
    \item $n=0$; $f(0)=0<1=\left(\frac{5}{3}\right)^{0}$, de donde sabemos que $P(0)$ vale.
    \item $n=1$; $f(1)=1<\frac{5}{3}=\left(\frac{5}{3}\right)^{1}$, de
      donde sabemos que $P(1)$ vale.
    \item $n>1$; considere las siguientes realciones:
      \begin{align*}
        f(n)&=f(n-1)+f(n-2)&& (\text{por definición})\\
            &<\left(\frac{5}{3}\right)^{n-1}+\left(\frac{5}{3}\right)^{n-2}&&(\text{por hip. induc.})\\
            &=\left(\frac{5}{3}\right)^{n-2}(\frac{5}{3}+1)&&\\
            &=\left(\frac{5}{3}\right)^{n-2}\frac{8}{3}&&\\
            &<\left(\frac{5}{3}\right)^{n-2}\left(\frac{5}{3}\right)^{2}&&(\text{pues }72<75)\\
            &=\left(\frac{5}{3}\right)^{n}&&\\            
      \end{align*}
    \end{itemize}
    Por el \textit{Segundo Principio de Inducción Finita} sabemos que
    para todo número natural $n$, vale $P(n)$.
  \end{solution}