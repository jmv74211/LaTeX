\documentclass[10pt]{article}

% a4large.sty -- fill an A4 (210mm x 297mm) page
% Note: 1 inch = 25.4 mm = 72.27 pt
%       1 pt   =  3.5 mm (approx)

\usepackage[paperheight=297mm,
            paperwidth=210mm,
            tmargin=28mm, 
            headheight=0mm, 
            headsep=0mm,
            textheight=240mm,
            footskip=7mm,
            textwidth=159.2mm,
            bindingoffset=10mm,
            twoside
            ]{geometry}

% paragraph setup
\setlength{\parindent}{0em}
%\setlength{\parindent}{4em}
\setlength{\parskip}{1em}
\setlength{\columnsep}{1em}
%\renewcommand{\baselinestretch}{1.5}

\usepackage[utf8]{inputenc}
\usepackage[%english
            spanish
           ]
           {babel}
\usepackage[usenames,dvipsnames]{color}
%\usepackage{chessboard}
\usepackage[pdftex]{graphicx}
\usepackage{amsmath,amsthm,latexsym}
\usepackage{amsfonts,amssymb,MnSymbol}
\usepackage{enumerate}
\usepackage{subfigure}

\usepackage[bookmarks=true,
            bookmarksnumbered=false, % true means bookmarks in 
                                     % left window are numbered                         
            bookmarksopen=false,     % true means only level 1
                                     % are displayed.
            colorlinks=true,
            linkcolor=webred,
            urlcolor=NavyBlue,
            citecolor=OliveGreen
            ]{hyperref}
\definecolor{webgreen}{rgb}{0, 0.5, 0} % less intense green
\definecolor{webblue}{rgb}{0, 0, 0.5}  % less intense blue
\definecolor{webred}{rgb}{0.5, 0, 0}   % less intense red
\definecolor{dblackcolor}{rgb}{0.0,0.0,0.0}
\definecolor{dbluecolor}{rgb}{.01,.02,0.7}
\definecolor{dredcolor}{rgb}{0.8,0,0}
\definecolor{dgraycolor}{rgb}{0.30,0.3,0.30}

\newcommand*{\textcal}[1]{
             % family qzc: Font TeX Gyre Chorus (package tgchorus)
             % family pzc: Font Zapf Chancery (package chancery)
             \textit{\fontfamily{pzc}\selectfont#1}
                         }

\renewcommand{\labelitemi}{---)}
\renewcommand{\labelitemii}{$\ast$}
%\renewcommand{\labelitemii}{$\cdot$}
\renewcommand{\labelitemiii}{$\diamond$}
\renewcommand{\labelitemiv}{$\bullet$}

%%%%%%%%%% English %%%%%%%%%

\newtheorem{theorem}{Theorem}[section]
\newtheorem{corollary}[theorem]{Corollary}
\newtheorem{lemma}[theorem]{Lemma}
\newtheorem{proposition}[theorem]{Proposition}
\newtheorem{example}[theorem]{Ejemplo}
\newtheorem{ax}{Axiom}
\newtheorem{algoritmo}[theorem]{Algorithm}

\theoremstyle{definition}
\newtheorem{definition}{Definition}[section]

\theoremstyle{remark}
\newtheorem{remark}{Remark}[section]
\newtheorem*{notation}{Notación}

%\numberwithin{equation}{section}

%%%%%%%%%%%%%%%%%%%%%%%%%%%%%%%%%

%%%%% Para cambiar el tipo de letra en el título de la sección %%%%%%%%%%%
\usepackage{sectsty}
\chapterfont{\fontfamily{pag}\selectfont} %% for chapter if you want
\sectionfont{\fontfamily{pag}\selectfont}
\subsectionfont{\fontfamily{pag}\selectfont} 
            %% Similarly for others. see the manual of sectsty, section 5.
%%%%%%%%%%%%%%%%%%%%%%%%%%%%%%%%%%%%%%%%%%%%%%%%%%%%%%%%%%%%%%%%%%%%%%%%%%
\begin{document}

\setcounter{section}{-1}

\bibliographystyle{plain}


\title{\huge\bf El Vehículo Oruga}
\author{\large Antonio Carraro\\[1mm]
 Padua\\
 Italia}
\date{\today}
%\date{30/02/2100}
\maketitle

\begin{abstract}
  \noindent
  Recopilación de datos sobre tractores oruga para ilustrar el estilo
  ``article'' de \LaTeX\ centrándose en los tractores de oruga
  \textcal{Lamborghini} y \usefont{T1}{pzc}{m}{n} Carraro.
\end{abstract}
\tableofcontents % Hace el índice de contenidos.
\listoffigures
%%% Local Variables: 
%%% mode: latex
%%% TeX-master: "bbp_formula"
%%% End:

\section{Introducción}

La \textit{fórmula de Bailey-Borwein-Plouffe} (o \textit{fórmula BBP})
permite calcular el enésimo dígito de $\pi$ en base $2$ ó $16$ sin
necesidad de hallar los precedentes, de una manera rápida y utilizando
muy poco espacio de memoria en la computadora. \textit{Simon Plouffe}
junto con \textit{David Bailey} y \textit{Peter Borwein} hallaron esta
fórmula el 19 de septiembre de 1995 usando un programa informático
llamado \texttt{PSLQ} que busca relaciones entre números enteros.

La fórmula BBP tiene la siguiente expresión:
\begin{equation*}
  \pi = \sum_{k=0}^{\infty}\frac{1}{16^{k}}
  \left(
    \frac{4}{8k+1}
    -\frac{2}{8k+4}
    -\frac{1}{8k+5}
    -\frac{1}{8k+6}
  \right)
\end{equation*}
La demostración de esta igualdad se encuentra más abajo.
\input{tractor_oruga}

\nocite{let}

\section*{Referencias Web}

\begin{itemize}
\item \htmladdnormallink{Carro de combate Leclerc}
                  {http://es.wikipedia.org/wiki/AMX-56_Leclerc}

\item \htmladdnormallink{Tractores Carraro}
    {http://www.antoniocarraro.it/es}

\item \htmladdnormallink{La web de
      ritchiewiki}{http://www.es.ritchiewiki.com/wikies/index.php/Tractor_sobre_orugas}
\end{itemize}

\bibliography{oruga}

\end{document}