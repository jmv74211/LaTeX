         %%% bismi al-ll_ahi ar-ra.hmAni ar-ra.hImi %%%

\documentclass[10pt,spanish]{article}

% aprovechamiento de la p\'agina -- fill an A4 (210mm x 297mm) page
% Note: 1 inch = 25.4 mm = 72.27 pt
% 1 pt = 3.5 mm (approx)

% vertical page layout -- one inch margin top and bottom
\topmargin      -10 mm   % top margin less 1 inch
\headheight       0 mm   % height of box containing the head
\headsep          0 mm   % space between the head and the body of the page
\textheight     240 mm   % the height of text on the page
\footskip         7 mm   % distance from bottom of body to bottom of foot

% horizontal page layout -- one inch margin each side
\oddsidemargin    0 mm     % inner margin less one inch on odd pages
\evensidemargin   0 mm     % inner margin less one inch on even pages
\textwidth      159 mm     % normal width of text on page

\usepackage[utf8]{inputenc}
\usepackage[%english
            spanish
           ]
           {babel}
\usepackage[usenames,dvipsnames]{color}
\definecolor{RojoAnayelRey}{rgb}{1,.25,.25}
\usepackage[pdftex]{graphicx}
\usepackage{amsmath,amsthm,mathtools}
\usepackage{amsfonts,amssymb,latexsym,MnSymbol}
\usepackage{pifont}
\usepackage[active]{srcltx}
\usepackage[pdftex]{graphicx}
\usepackage{multirow}
\usepackage{enumerate}
\usepackage{subfigure}
%\usepackage{circuitikz}
%\usepackage{karnaugh-map}

%\usepackage[math]{iwona}
\usepackage{beton}
\usepackage[T1]{fontenc}

\newcommand{\regla}[2]{
\begin{array}{c}
#1\\
\hline
#2\\
\end{array}
}

\newenvironment{solution}{\begin{proof}[Solution]}{\end{proof}}

\begin{document}
%\color{RojoAnayelRey}
%\color{blue}

\begin{flushleft}
\hrule\vspace{4mm}
\textsc{Apellidos:}\makebox[115mm]\dotfill\ \textsc{Grupo:}\dotfill\\[4mm]
\textsc{Nombre:}\makebox[70mm]\dotfill\
\textsc{nif:}\makebox[40mm]\dotfill\ \textsc{nº hojas:}\dotfill\\[5mm]
\hrule
\end{flushleft}

\begin{center}
  \textbf{\huge LMD}\\[2mm]
  \textbf{\Large Grado en Ingeniería Informática y Matemáticas}\\[2mm]
  \textbf{\large 24 de enero de 2018}
\end{center}

\begin{flushleft}
  \begin{enumerate}
  \item Sea $e$ la función de argumentos naturales dada por:
    \begin{align*}
      e(a,0)&=1,\\
      e(a,b)&=
              \begin{cases}
                e\left(a^{2},\frac{b}{2}\right),&\text{ si $b$ es par.}\\
                e\left(a^{2},\frac{b-1}{2}\right)a,&\text{ si $b$ es impar.}
              \end{cases}
    \end{align*}
    Demuestre por inducción que para cualesquiera números naturales
    $a$ y $b$, $e(a,b)=a^{b}$.
    
  \item Resuelva el problema de recurrencia:
    % Veerarajan, pag. 355; pero tiene un error
    \begin{equation*}
      u_{n+2}-6u_{n+1}+9u_{n}=3\cdot 2^{n}+7\cdot 3^{n}, \quad n\geq 0.
    \end{equation*}
    y encuentre la solución particular que cumple $u_{0}=1$ y $u_{1}=4$.

\item Demuestre que para todo conjunto de fórmulas proposicionales
  $\Gamma\cup\{\alpha,\beta\}$ se cumple:
  \begin{enumerate}
  \item $\operatorname{Con}(\Gamma\cup\{\alpha\wedge\beta\})=
    \operatorname{Con}(\Gamma\cup\{\alpha,\beta\})$
  \item $\operatorname{Con}(\Gamma\cup\{\alpha\vee\beta\})=
    \operatorname{Con}(\Gamma\cup\{\alpha\})\cap \operatorname{Con}(\Gamma\cup\{\beta\})$
  \end{enumerate}

\item Estudie si el siguiente conjunto de cláusulas:
  \begin{align*}
    \{&\neg a\vee c\vee f, b\vee c\vee f, b\vee\neg c\vee f,
       \neg b\vee f, a\vee\neg b\vee f, \neg a\vee d\vee f, d\vee f,\\ 
      &b\vee d\vee e\vee\neg f, b\vee\neg d\vee e\vee\neg f, b\vee\neg e\vee\neg f,
      a\vee\neg b\vee\neg c\vee\neg f, \neg a\vee\neg b\vee c\vee\neg f\}
  \end{align*}
  es o no insatisfacible y caso de ser satisfacible, de una asignación
  que lo evidencie.

\item Considere la función booleana de cuatro variables:
  % http://www.eecg.utoronto.ca/~jayar/ece241_06F/solved4.pdf
  \begin{equation*}
    f(a,b,c,d)=\bar{a}\bar{c}\bar{d}+cd+\bar{a}\bar{b}d+ab\bar{c}d
  \end{equation*}
  Si supone que hay también términos ``no importa'' definidos por
  $D(a,b,c,d)=\sum d(9,12,14)$, dé para $f$:
  \begin{enumerate}
  \item una descomposición minimal como suma de productos,
  \item una descomposición minimal como producto de sumas y
  \item al menos una expresión que mejore el costo de cualquiera de
    las dos anteriores.
  \end{enumerate}

\item Considere las cuatro fórmulas cerradas siguientes:
  \begin{itemize}
  \item $\gamma_{1}\equiv \forall x\forall y (r(x,y)\to r(y,x))$
  \item $\gamma_{2}\equiv \forall x\forall y\forall z (r(x,y)\wedge
    r(y,z)\to r(x,z))$
  \item $\gamma_{3}\equiv\forall x\exists yr(x,y)$
  \item $\varphi\equiv \forall xr(x,x)$
  \end{itemize}
  Responda razonadamente a las siguientes preguntas:
  \begin{enumerate}
  \item ¿Es $\varphi$ consecuencia lógica de $\{\gamma_{1},\gamma_{2}\}$?
  \item ¿Es $\varphi$ consecuencia lógica de $\{\gamma_{1},\gamma_{2},\gamma_{3}\}$?
  \end{enumerate}
  
\item Encuentre una fórmula en forma normal prenexa lógicamente
  equivalente a:
  \begin{equation*}
    \forall xp(x,y)\to\bigl(\forall
    yp(y,x)\to\forall x(q(x)\wedge\exists y\forall
    zr(a,y,z))\bigr)
  \end{equation*}
  y que tenga el mínimo número de cuantificadores. Seguidamente dé una
  forma normal de Skolem para esa forma normal prenexa antes hallada.

% \item Encuentre, si es posible, un unificador de máxima generalidad o
%   principal para el siguiente par de literales:
%   \begin{equation*}
%     \langle p(g(f(x),u),f(a),g(z,f(y))),p(g(f(f(y)),g(v,a)),f(v),g(g(x,b),x)\rangle
%   \end{equation*}
%   % flpjun0809.pdf preg. 7

\item Demuestre, usando resolución lineal input, que la
  fórmula:
  \begin{equation*}
    \neg\exists x(r(x)\wedge s(x)) 
  \end{equation*}
  es consecuencia de las fórmulas:
  \begin{itemize}
  \item $\forall x((r(x)\wedge s(x))\to\exists y(q(y)\wedge p(x,y)))$
  \item $\forall x(q(x)\to t(x))$
  \item $\forall x(t(x)\to o(x))$
  \item $\forall x\forall y((r(x)\wedge o(y))\to\neg p(x,y))$
  \end{itemize}
\item Responda razonadamente a las siguientes preguntas:
  \begin{enumerate}
  \item Demuestre que un árbol finito $G$ (con al menos un vértice)
    tiene al menos dos vértices de grado $1$. 
    %Lipschutz-Lipson 2000 problemas. 5.137 pag. 185
  \item Halle el número $m$ de aristas de los grafos: $K_{8}$,
    $K_{12}$ y $K_{15}$. ¿Cuál es el diámetro de $K_{n}$? %5.115
  \item Dé un grafo de $6$ vértices que sea hamiltoniano pero no
    euleriano. Dé su matriz de adyacencia. %5.105  
  \end{enumerate}
\end{enumerate}
\end{flushleft}

\end{document} 